\documentclass{article}
\usepackage{graphicx}
\usepackage{authblk}
\usepackage{multicol}
\usepackage{algorithm}
\usepackage{algpseudocode}
\setlength{\columnsep}{0.5cm}

\title{Joint Optimization of Trajectory Planning and Task Scheduling in
Heterogeneous Multi-UAV System}
\author{Shubham Kapoor}
\affil{Department of Computer Scince, IIT Ropar}
\begin{document}

\maketitle
\begin{multicols}{2}
\section*{Abstract}
The use of unmanned aerial vehicles (UAV) as a new sensing
paradigm is emerging for surveillance and tracking
applications, especially in the infrastructure-less environment.
One such application of UAVs is in the construction
industry where currently prevalent manual progress
tracking results in schedule delays and cost overruns. In
this paper, we develop a heterogeneous multi-UAV framework
for progress tracking of large construction sites. The
proposed framework consists of Edge UAV which coordinates
the data relay of the visual sensor-equipped Inspection
UAVs (I UAV s) to the cloud. Our framework jointly
takes into consideration the trajectory optimization of the
Edge UAV and the stability of system queues. In particular,
we develop a Distance and Access Latency Aware
Trajectory (DLAT) optimization that generates a fair access
schedule for I UAV s. In addition, a Lyapunov based
online optimization ensures the system stability of the average
queue backlogs for data offloading tasks. Through a
message based mechanism, the coordination between the
set of I UAV s and Edge UAV is ensured without any
dependence on any central entity or message broadcasts.
The performance of our proposed framework with joint
optimization algorithm is validated by extensive simulation
results in different parameter settings.\\
Keyword: Path Planning, Task Scheduling, Data Offloading,
Construction Site Monitoring, Unmanned Aerial
Vehicles (UAVs), Lyapunov Optimization

\section{Introduction}
The unmanned aerial vehicle (UAV) based solutions are
emerging in various domains such as wireless sensing \cite{mozaffari2019tutorial},
payload delivery [2], precision agriculture [3], help and
rescue operations [4], etc. Moreover, with the current
trend of automation, sensing and information exchange in Industry 4.0, UAV based applications are also finding
their place in the construction industry especially for resource
tracking and progress monitoring using aerial imagery.
Such solutions are helpful in infrastructure-less
large construction sites as they provide ease of deployment,
quick access to the ground-truth data and higher
reachability and coverage [5]. Further, the autonomous
or semi-autonomous UAV based solutions could facilitate
progress monitoring, building inspections (for cracks or
other defects), safety inspections (to find any environmental
hazards) and many more construction-specific audits
automatically. The UAV based visual monitoring of
under-construction projects also allows simultaneous observability
of ground-truth data by different collaborating
entities. Availability of such data and information helps
in timely assessments that could reduce schedule delays,
cost overruns, resource wastage and financial losses which
are not uncommon in construction projects.
A plausible solution to address the aforementioned
challenges could be a Mobile Edge Computing (MEC)
[6] based heterogeneous multi-UAV framework. Such a
framework along with the prior geometric knowledge available
about the construction site as gathered from a Building
Information Model (BIM)[7] could help create an effective
multi-UAV based visual monitoring system for construction
sites. As for any constrained environment, the
optimization of computational resources is central to develop
a solution. The integration of UAVs and MEC into
a single framework could facilitate that with efficient data
collection/processing from the UAV based dynamic sensors
in infrastructure-less environments [8]. In addition,
an MEC based framework can help to perform partial
computation offloading wherein a part of data is processed
by the UAVs while the rest gets offloaded to the cloud.
An MEC based UAV framework is not new and the deployment
of the UAVs as base stations or edge servers
is widely studied [9, 10]. These studies reflect on the flexibility in deployment of UAV based edge computing
components. However, there is a problem of buffer overflow
of UAVs due to the limited on-board processing and
the shared bandwidth to transfer data to the cloud which
leads to instability in the system. In addition, the dynamic
nature of such systems with varying data traffic
and continuous movement of UAVs makes it difficult to
stabilize or control the system in a deterministic manner.
Researchers have used online Lyapunov optimization
[11] to address such system instabilities. Lyapunov optimization
considers the stability of the system with time
varying data and optimizes time averages of system utility
and queue backlogs.
In this paper, we address the challenges of deploying
a heterogeneous multi-UAV system for construction site
monitoring by the joint optimization of UAV trajectory
planning and data offloading task scheduling. The proposed
framework employs two types of UAVs viz. Inspection
UAVs I UAV s and Mobile Edge UAVs (Edge UAV ).
While the former is deployed as visual sensors to collect
visual data from different locations of the site, the latter
interacts and collects data from I UAV s, and offloads the
same to the cloud. The core objective of the framework
is to minimize the total energy consumption of the system
while considering the data queue backlogs of I UAV s
and Edge UAV and also jointly optimizing the trajectory
of the Edge UAV in accordance with the trajectories of
I UAV s having minimum access latency and travel distance.
The online resource management such as transmission
power and processor frequency of the Edge UAV is
evolved using Lyapunov optimization (as in [12]).
The rest of the paper is organised as follows: Section
2 presents the proposed heterogeneous multi-UAV framework
for construction site monitoring. The overall system
objective is discussed in Sections 3. Sections 4 and 5 discuss
the trajectory optimization and Lyapunov based system
stability, respectively. The simulation setup has been
presented in Section 6. Section 7 discusses the results
gathered from the experiments while Section 8 concludes
the paper.

\section{Heterogeneous Multi-UAV Framework}
Figure ?? depicts the overall multi-UAV framework with
all its components. The system consists of two heterogeneous
UAVs i.e. a set of Inspection UAVs I UAV =
{I UAV1, I UAV2, I UAV3, ....., I UAVN} and a Mobile
Edge UAV (Edge UAV ). I UAV s are smaller in size and
are more agile. They collect visual data from a set of
Point of Interests (PoIs) denoted as L = {l1, l2, l3....lk}
across the construction site. As the construction sites are
infrastructure-less environments, there are limited Access
Points (AP) available for connectivity to the cloud. Further,
the I UAV s possess limited connectivity range that
makes it difficult for them to transfer data to cloud directly.
In addition, the I UAV s move in the 3D Cartesian
coordinate system. The Edge UAV , which is larger in size
and possesses higher computational capabilities, coordinates
with the I UAV s to relay the data (after partially
processing the same) to the cloud. Edge UAV always
maintains a constant height and thus its trajectory lies in
an horizontal plane.
The communication between I UAV and Edge UAV
(A2A channel) has limited range and bandwidth. We
have assumed the achievable data transmission rate of
the I UAVi in a given time slot as doff
i (t). Further, The
height of the Edge UAV is h which is dependent on coverage
range r and line of sight (LoS) loss caused due to
environmental effects [13].
The A2A channel power gain $\zeta$ from I UAV to
Edge UAV can be given as: 
\begin{equation}
    \zeta = g_o * (dis_0/dis_t)^\theta
\end{equation}
where g0 is the path loss constant, dis0 is the reference
distance, dist distance between the UAVs, and $\theta$ is the
path loss exponent.
\subsection{Data collection and offloading}
Each PoI $l_i$ is a tuple $(< d_i, v_i>)$ where $ d_i$ specifies the
amount of data (images) to be collected and $v_i$ denotes the
coordinates of the site locations in 3D space. The sequence
of PoIs to be visited is provided to I UAV s and same is
also shared with the Edge UAV . During the traversal
along the sequence of PoIs, the limited buffer may make
the I UAV wait at some PoIs along the trajectory until
it offloads the data to the Edge UAV .
The Edge UAV can communicate with one of the
I UAVi in a time slot. The data gathered by each of
the I UAVi in a time slot t is denoted by Ai(t). Qi(t)
represents the queue of the I UAVi and doff
i (t) denotes the amount of data offloaded to the Edge UAV by the
I UAVi in time-slot t. The recursive equation to update
the Qi(t) is as follows: 
\begin{equation}
    Q_i(t+1)=max{q_i(t)-d_i^{off}(t),0}+A_i(t)
\end{equation}
The Edge UAV accepts data from the selected I UAVi
in the time-slot t in its queue L(t). The following equation
updates L(t) recursively:
\begin{equation}
    L(t+1)=max{L(t)-c(t)-d_{edge}^{off}(t),0+A_egde(t)}
\end{equation}
where $A_edge(t)$ is the data arrived from the selected
I UAVi in time-slot t, c(t) is the data processed by the
Edge UAV in time-slot t, and $d_{edge}^{off}$ is the number of
bits offloaded to the cloud in time-slot t.

\section{System Objective}
In the proposed framework, the offloading of data happens
at two stages - 1) from I UAVi to Edge UAV and 2) from
Edge UAV to the cloud. Our main focus is to achieve the
end-to-end data offloading to the cloud by minimizing the
total energy consumption of the whole system (Esystem)
which is defined as:
\begin{equation}
    E_{system}(t)=E_{edge}^{transition}(t)+E_{edge}^{Comm}(t)+(\sum_{i=1}^N(E_i^{Comm}(t)))
\end{equation}
where Etransition
edge (t) is the transition energy of the
Edge UAV , EComm
edge (t) is a communication energy of the
Edge UAV and EComm
i (t) is the communication energy
of the ithI UAVi.
Further, we discuss the various components of Esystem
along with the expressions to calculate the same.

\subsubsection{Transition energy of Edge UAV}
The transition energy of Edge UAV refers to the energy
consumed in moving from one location to another. The
transition energy of the Edge UAV is given as:
\begin{equation}
    E_{edge}^{transition}=K(vel(t))^2
\end{equation}

where K is a constant that depends on the total mass of
the Edge UAV and vel(t) is the velocity of I UAV 

\subsubsection{Communication energy of Edge UAV}
Edge UAV offloads the data to cloud through a wireless
channel [14]. The communication energy consumed
to transmit the data to the cloud is given as:
\begin{equation}
        E_i^{comm}(t)=(2^{d_i^{off}(t)/{W*r}}-1)
\end{equation}
where the parameters are defined in the Table ??

\subsubsection{Communication Energy of I UAV}
The energy consumed for offloading the doff
i (t) data bits
at time slot t from the selected I UAVi to the Edge UAV
using the A2A channel of bandwidth W Hz is given similarly
to Equation 6 as:
% \begin{equation}
    
% \end{equation}
As the PoIs are predefined and the I UAV s follow a
predetermined path, the energy consumed for the movement
of I UAV s are not taken into consideration.
Given the energy of the system, our goal is to find
the optimal parameter values so as to minimize the expected
cumulative energy across the time horizon. The
system policy in every time-slot t can be given by X(t) =
{Fedge(t), pi(t), Pedge(t), Sedge(t)}. Hence, the end-to-end
data offloading policy parameters X(t) aims at minimizing
total expected energy of the system. As the channel
information for the data offloading is not deterministic
and varies in the environment, the amount of bits
arrived at the Edge UAV depends upon the channel characteristics
as well as the current position of the selected
I UAVi. Such time-coupling of variables is responsible for
the stochastic nature of the system. The overall optimization
model for the stable system performance is given as:


The constraints C1 and C3 defines the maximum
frequency and maximum transmission power of the
Edge UAV respectively. In addition, C5 defines the maximum
number of bits processed by Edge UAV . Furthermore,
C4 and C6 upper bound the number of transmitted
bits. Similarly, for I UAV , the constraints C2, C4 and C6
bound the number of transmitted bits. The constraints
C8 and C9 establish the rate stability of all system queues
(I UAVi and Edge UAV ). Next we discuss the model to
optimize the trajectory of the Edge UAV with respect to
the trajectories of I UAV s.

\section{Distance and Latency Aware Trajectory}
The flexible and dynamic trajectory planning of
Edge UAV could help in applications within construction
industry where it is hard to reach by terrestrial communication
infrastructure. As already mentioned, the position
of I UAV s changes in every time-slot since they
move through different PoIs to collect data. Hence, the
Edge UAV ’s trajectory needs to be estimated in such a
manner that it can connect and access an I UAVi in a
time-slot before the I UAVi’s queue overflows. Whenever
an I UAVi’s queue gets full, it doesn’t move to its next
designated PoI and sojourns at the same PoI until it is
able to offload its data to the Edge UAV and free up
some queue space. Hence, in order to choose one of the
I UAV s to offload its queue, the Edge UAV would require
the real-time information about the queues of all the
I UAV s in each time-slot. This information is not available
a priori due to the dynamic nature of the system. We
use a message passing based approach for estimating the
queue sizes of the I UAV s in order to make a selection.
Further, the trajectory of the Edge UAV must be optimized
so as to consume minimal energy. The trajectory
optimization model of Edge UAV optimizes the trade-off
between transition energy of Edge UAV and access latencies
of all I UAVis. In addition, the access latency based
data offloading generates a fair schedule for the I UAV s
to offload data to the Edge UAV . Access latency (Ri(t))
of the ith I UAVi in the time-slot t is the difference between
the time of its last access by the Edge UAV and
the current time-slot.
% \begin{equation}
    
% \end{equation}
The first constraint C1 of optimization model P2 signifies
the distance travelled within a time-slot is limited by the
maximum velocity. The following constraint C2 restricts
that the selected I UAVi should be in the coverage range
of the Edge UAV . Constraint C3 denotes that the queue
of the selected I UAVi shouldn’t be empty while C4 limits
the time that has elapsed since the last access of ith
I UAVi should be less than Rmax. The constraint in C5
selects the I UAVi which has data to offload whereas C6 is a binary constraint to select only one of the I UAVi in
a time-slot.

\section{Lyapunov Optimization based System
Stability}
The model presented in P1 in Section 3 is a stochastic optimization
problem. The data arrival at the system queues
is random in nature. With the help of online Lyapunov
optimization algorithm, we can solve such stochastic optimization
models and jointly stabilize all queues by finding
the optimal X(t) in each time slot [15].
The quadratic Lyapunov function [15] associates a
scalar measure to queues of the system. Further, the stability
of the system is maintained by a guaranteed mean
rate stability of the evolving queues i.e. 

$ $ consists of all backlog queues of
the system at time t and Z(.) is quadratic Lyapunov function
of system queues.
The Lyapunov drift corresponding to above function
can be given as:

The Lyapunov drift plus a penalty function is minimized
to stabilize the queue backlog of the system which is given
as:

where V is a positive system constant which controls the
trade-off between Lyapunov drift and the expected energy
of the system. A high value of parameter V signifies more
weight on minimizing energy of the system at the cost of
high queue backlog. Hence, V acts as a trade-off parameter
between system’s energy and queue backlog.
An upper bound on $\delta(v(t))$ can be derived as, (for
details see [15, 11])

where C is a deterministic constant.
As a result, the upper bound of the drift plus penalty
function becomes

Hence, the original formulation P1 is reduced to P3 which
bounds the system’s drift to keep the system stable as
follows:
\[\]
As can be observed, the constraints in P3 is a subset of
the constraints in P1. To further simplify the solution of
the optimization formulation, P3 could be reformulated
as two separate sub-problems provided the positions of
Edge UAV and I UAVi are fixed in a given time slot t.
\subsubsection{Transmission energy optimization of
I UAV s}
First sub-problem deals with the optimization of parameters
related to the I UAVi. The variables Sedge(t) i.e. position of Edge UAV and the offloaded bits of the selected
I UAVi are coupled in particular time interval. The
fixed position of Edge UAV decouples these variables. In
the optimization model P 3.1, the transmission energy
is optimized for a single time-slot given the position of
Edge UAV : 
\[\]
It can be observed that objective function in P 3.1 is a
convex function. First constraint is linear and the second
constraint is upper bounded by a concave function.
As a result, the stationary point of the objective function
can be derived as: $p_i^*(t)=min{max{N_o/c((Q_1(t)W)/V)+1)}}$

\subsubsection{Transmission energy optimization of
Edge UAV}
The second sub-problem deals with the optimization of the
Edge UAV parameters for the amount of data offloaded
to the cloud. Further, here we can ignore the processor
frequency parameters and the associated constraints from
the optimization as they do not affect the energy optimization.
The updated optimization model is given as:
\[\]
The model P 3.2 has a convex optimization objective subject
to convex constraints to solve the optimal transmission
power of the Edge UAV . The stationary point of the
optimization model P 3.2 is $P_edge(t)=N_o/c((L(t)Wr)/V-1)$
The overall solution approach of the proposed heterogeneous
multi-UAV framework is given in Algorithm 1.


\bibliographystyle{plane}
\bibliography{references}

\end{document}
